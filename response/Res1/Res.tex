\documentclass[aps,article,author-year,notitlepage,showpacs]{revtex4-1}
%\documentclass[a4paper]{article}

\usepackage{color}
\usepackage{enumerate}
\usepackage{amssymb,amsmath}
\usepackage{bm}% bold math
\usepackage{bbold}


\begin{document} 

\large

\noindent {\bf DW12052 Wen}\\

\noindent authors: Shi Yin, Rui Wen, and Wei-jie Fu\\

\noindent We thank the referee for his comments and suggestions, which we have taken into account in the revised version:\\

\noindent \underline{Major changes}:\\

We agree with the referee's major comment on our paper in its first version, i.e., in the referee's word, ``{\it Certainly the authors carried out a new calculations with a sophisticated framework, but I wonder whether this paper presents new physics inherent to the novel features of the computations}''. Following his suggestion, we have performed more computations and further studies, and focused on the role of strangeness degrees of freedom and its fluctuations. Furthermore, we have also investigated the effects of the mesonic fluctuations in the 2+1 flavor effective model within FRG. Relevant new results are presented in the revised version as follows

\begin{enumerate}[1.]

\item In the revised version, we have extended the only one baryon chemical potential $\mu_B$ in the old version to a set of chemical potentials for different flavors, i.e., $\mu_u=\mu_B/3+2\mu_Q/3$, $\mu_d=\mu_B/3-\mu_Q/3$, and $\mu_s=\mu_B/3-\mu_Q/3-\mu_S$, which facilitates investigations of the fluctuations for different flavors and conserved charges. All the related formulas, such as the flow equation in Eq.(16), are modified. Note that while $\mu_B$ does not contribute to the mesonic loops in the flow equation, other chemical potentials indeed do, e.g., $\mu_s$ for open Kaons.\\[-3ex] 

\item We have investigated the fluctuation of the strangeness and its dependence on the quantum fluctuations of mesons, especially the Kaons. The relevant results are presented in Fig.5 in the revised version, where we also compare the fluctuations between the light and strange quarks. Relevant descriptions and discussions are presented in a paragraph beginning with ``In Fig.5, we show the fluctuations of the light and ......'' on page 8.\\[-3ex] 

\item We have extended the highest order of the generalized susceptibilities from the quartic to the sixth, which allows us to compare our calculated $\chi^{B}_6/\chi^{B}_2$ with the lattice result in the right panel of Fig.4. In Fig.5 we also present relevant results of the sixth order.\\[1ex]

\end{enumerate}


\noindent \underline{Response to the minor issues}:\\

\begin{enumerate}[1.]

\item {\it The title of this paper contains low energy effective ``theory'', whereas the authors mostly use ``model'' in the text. I think that the latter is more appropriate at present.} \\[0.3ex] 
 
We have replaced the ``theory'' with the ``model''. \\[0.3ex] 

\item {\it There seem to be some biases in the references. For instance, there are a number of works concerning CEP in the same class of models and its phenomenology such as Asakawa and Yazaki ('89) and Stephanov and others ('99) and so on, while the authors quote only [41]-[44].} \\[0.3ex] 

In the context of the Refs. [41]-[44], we emphasize the importance of the baryon number fluctuations for the search of the CEP, not just the CEP in the effective model. Therefore we think that the work by Stephanov et al. 1999 is relevant and is cited in the revised version, while that by Asakawa and Yazaki 1989 is not.\\[0.3ex] 

\item {\it Actually the present work does not address CEP at all, while it is stated that this work was motivated by CEP in the introduction. Clearly the Taylor expansion-based formulation is not suitable for CEP and connecting first order phase transition, thus I would suggest the authors to rephrase related paragraphs to correctly reflect the contents.}\\[0.3ex] 

We do not agree with the referee. First of all, The CEP is addressed in this work. We have investigated the phase structure and a phase diagram is presented in Fig.6. The location of the CEP is predicted within the low energy effective model. Although the Taylor expansion is not suitable for the CEP and the first-order phase transition, the CEP is still accessible through this approach. Furthermore, this work is focused on the fluctuations of the baryon number or conserved charges, which is closely related to the CEP and the experiments aiming at the search of the CEP. Therefore, we think it is not in conflict with the main text to stress the CEP in the introduction. But, we would like to thank the referee's suggestion.

\item {\it In the introduction to net-baryon number cumulants, probability distribution $P(N_B)$ is mentioned while it is not actually used. It seems to exist only to cite their own work [62]? If the authors stick to discuss $P(N_B)$, they should expand it in more details because there are practical differences. (See, Morita at al PRC`13). If not, it should be omitted to shorten the manuscript length.}\\[0.3ex] 

We do not agree with the referee. When the net baryon number is discussed, it is more naturally to introduce the probability distribution. Since Eq.(21) can be regarded as the definition of the cumulants. It has been verified in our work (arXiv:1805.12025) that the probability distribution and the generalized susceptibility give the same results, when the Polyakov loop is taken into account, rather than different, so it is {\bf not} purposed to cite our own work in the context.

\item {\it The various ratios of the net-baryon number cumulants have been introduced by Ejiri et al ('06) and Karsch et al ('11), in particular its relevance to the degrees of freedom was pointed out by the former. These references should be included.}\\[0.3ex] 

Thank the referee for pointing out these references, which have been cited in the revised version.


\end{enumerate}




















\end{document}
