\documentclass[aps,article,author-year,notitlepage,showpacs]{revtex4-1}
%\documentclass[a4paper]{article}

\usepackage{color}
\usepackage{enumerate}
\usepackage{amssymb,amsmath}
\usepackage{bm}% bold math
\usepackage{bbold}


\begin{document} 

\large

\noindent {\bf DG12856 Yin}\\

\noindent authors: Shi Yin, Rui Wen, and Wei-jie Fu\\

\noindent We are grateful to the referee for his comments and suggestions, which we have taken into account in the revised version:\\[0.3ex] 

The referee's comment on the influence of the splitting effect on thermodynamical observables are interesting, and we agree with it. So we modify the relevant statements in Sec. IV A and Sec. V.\\[0.3ex] 

\noindent \underline{Response to the physics questions}:\\

\begin{enumerate}[1.]

\item We do the calculation again with the referee's suggestion ,``{\it Alternatively, it would also be good if the authors can show the deviations in thermal quantities if the splitting is artificially increased (say 10 x )}''. We take 10 times flow equation of the $Z^\|_\phi$ in our calculation and investigate the thermal quantities. We find that, it influences the values of the thermal quantities at low temperature, and affects the baryon number fluctuations at high temperature.\\[0.3ex] 

\item We take the referee's recommendation ``{\it For a meaningful comparison, I think it would be useful to single out the contribution from the φ part, e.g. the quasi-particle pressure due to φ, with and without splitting.}''. We calculate the pressure of the system only take the meson part into account in the flow equation of the effective potential. We can see from the result, the difference between with and without splitting of the $Z_\phi$ is not too much.
\end{enumerate}


\noindent \underline{Response to the questions concerning the text}:\\

\begin{enumerate}[1.]

\item {\it LPA$'$ needs to be defined explicitly. } \\[0.3ex] 
 
LPA$'$ is the truncation under which $\partial_tZ_{\phi/q}\neq 0$ and $\partial_th=0$. We have added the definition of the LPA$'$ in the paper.\\[0.3ex] 

\item {\it Can the authors justify why $Z_{\sigma} = Z_{\pi}$, while their mass functions are very different?} \\[0.3ex] 

In the context \\[0.3ex] 

\item {\it Fig.1: Please clarify whether all the Z's at UV point are the same, and what are the values?}\\[0.3ex] 

Yes, the value of the wave function renormalizations at UV point is equal, they are all set to 1. The value of the $Z_{UV}$ doesn't influence the behavior of the flow of the $Z$. \\[0.3ex]

\item {\it Fig.1: Please comment on whether there are interesting differences at finite $\mu$.}\\[0.3ex] 

The value of the difference between the $Z^{\|}_{\phi}$ and $Z^{\bot}_{\phi}$ at $k=0$ under finite $\mu_B$ is similar to the $\mu_B=0$, but both of the $Z$  turn smaller.

\item {\it I assume the Polyakov loop value is obtained from a mean field calculation similar
to the study in Ref. Phys.Rev.C83:054904,2011. Please clarify.}\\[0.3ex] 

Yes, the Polyakov loop value is obtained by the equation of motion $\partial_L\Omega=0$.

\item {\it p.6 right column: spacial (or spatial?)}\\[0.3ex] 

It is spatial, and we have modified it in the paper.

\item {\it Fig.8: At low temperature, the Polyakov loop is more appropriately approximated as 0 rather than 1. Can the authors comment on what happens to the scheme dependence there?}\\[0.3ex] 

Because in our calculation the fermion distribution function has the form $n_f(x,T,L,\bar{L})=\frac{1+2\bar{L}e^{x/T}+Le^{2x/T}}{1+3\bar{L}e^{x/T}+3Le^{2x/T}+e^{3x/T}}$. At low temperature we need the distribution function to be a baryonic distribution function. We can see in the function above, if the $L$ and $\bar{L}$ tend toward 0, the $n_F(x,T,L,\bar{L})$ become $\frac{1}{1+e^{3x/T}}$ which is a baryonic distribution function. More details see Ref. [34] Phys.Rev.D92,116006 (2015). 

\end{enumerate}




















\end{document}
